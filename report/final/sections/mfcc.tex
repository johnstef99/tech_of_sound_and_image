\subsection{Mel Frequency Cepstral Coefficients}
Ένα από τα χαρακτηριστικά που εξάγαμε από τα σήματα καρδιακών ήχων που είχαμε
στη διάθεση μας ώστε να εκπαιδεύσουμε το νευρωνικό δίκτυο είναι τα Mel Frequency
Cepstral Coefficients. Ο υπολογισμός τους έγινε στα ήδη προεπεργασμένα δεδομένα
και συγκεκριμένα στο κάθε ένα σήμα που δημιουργήθηκε από την έξοδο των
παραθύρων.Το συνόλο των συντελεστών που παρήγαγε αυτή η διαδικασία είναι 13 από
τους οποίους οι 12 αναπαριστούν την περισσότερη πληροφορία της φασματικής
περιβάλλουσας και ο 13\textsuperscript{ος} αναπαριστά τη συνολική ενέργεια του
σήματος. Δεν επιλέχθηκαν περισσότεροι από 13 συντελεστές καθώς η αύξηση του
αριθμού τους πάνω από αυτό το όριο έχει ως αποτέλεσμα την ταχεία μεταβολή των
συντελεστών γεγονός που δυσχαιρένει την εκπαίδευση του νευρωνικού δικτύου. Ο
κύριος λόγος που επιλέχθηκαν τα mfcc's είναι ότι αποτελούν την καλύτερη
προσέγγιση της λειτουργίας του κοχλία του ανθρώπινου αυτιού που είναι
επιλεκτικός στις συχνότητες και στο πως αντιδρά σε αυτές. Επιγραμματικά η
διαδικασία για τον υπολογισμό των mfcc είναι \cite{mfcc}

\begin{itemize}
	\item Εφαρμογή επικαλυπτόμενων παραθύρων στο σήμα
	\item Υπολογισμός του φάσματος ενέργειας
	\item Εφαρμογή του φίλτρου Mel και άθροισμα της ενέργειας του κάθε φίλτρου
	\item Λογαρίθμηση του αποτελέσματος του προηγούμενου βήματος
	\item Εφαρμογή μετασχηματισμού συνημιτόνων
\end{itemize}

\begin{figure}[H]
	\center
	\includegraphics[width=0.8\textwidth]{images/MelFilter.png}
	\caption{Φίλτρο Mel}
	\label{melfilter}
\end{figure}

Το δεύτερο χαρακτηριστικό που εξήχθει είναι το σπεκτόγραμμα mel που αποτελεί
μια μη γραμμική αναπαράσταση του συχνοτικού περιεχομένου του σήματος. Η
διαδικασίας υπολογισμού είναι παρόμοια με εκείνη για τα mfcc. Επιγραμματικά τα
βήματα που ακολουθούνται είναι \cite{melspectogram}

\begin{itemize}
	\item Εφαρμογή επικαλυπτόμενων παραθύρων στο σήμα
  \item Υπολογισμός του φάσματος ενέργειας μέσω του γρήγορου μετασχηματισμού
    Fourrier
	\item Μετατροπή του συχνοτικού περιεχομένου σε dB
	\item Εφαρμογή της κλίμακας mel στο συχνοτικό περιεχόμενο
\end{itemize}
 
\subsection{Eξαγωγή στην Python}
Για τις ανάγκες υλοποίησης του νευρωνικού δικτύου ήταν απαραίτητη η εξαγωγή των
mfcc's από τα φωνοκαρδιογραφήματα  που είχαμε στη διάθεσή μας. Μετά την
προεπεξεργασία στην οποία υποβλήθηκαν όταν πλέον είχαμε τα δείγματα μας
χωρισμένα σε μικρότερα από επικαλυπτόμενα παράθυρα τότε σε κάθε ένα καινούριο
ηχητικό σήμα το οποίο είχε δημιουργηθεί εφαρμόστηκε η συνάρτηση mfcc της
βιβλιοθήκης \verb|python_speech_features| η έξοδος της οποίας, τα 13 mfcc's,
αποθηκεύτηκαν σε μορφή εικόνας που είναι και τα χαρακτηριστικά εκπαίδευσης  του
συνελικτικού νευρωνικού δικτύου. Όσο αφορά το σπεκτόγραμμα mel χρησιμοποιήθηκε
από τη βιβλιοθήκη \verb|librosa.feature| η συνάρτηση \verb|melspectogram| που
πάλι η έξοδος του αποθηκεύεται σε μορφή εικόνας. Η συνάρτηση εφαρμόστηκε στα
ίδια ακριβώς σήματα με εκείνα που ήταν είσοδος της mfcc.
