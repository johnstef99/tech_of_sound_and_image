\documentclass{article}

\usepackage[LGR, T1]{fontenc}
\usepackage[utf8]{inputenc}
\usepackage[greek,english]{babel}
\usepackage[a4paper]{geometry}

\usepackage{alphabeta}
\usepackage{verbatim}
\usepackage{float}
\usepackage{hyperref}
\usepackage{indentfirst}
\usepackage{url}
\usepackage{chngcntr}
\counterwithin{figure}{section}

\usepackage{graphicx}
\graphicspath{{../images/}}

\AtBeginDocument{\renewcommand\refname{Αναφορές}}
\AtBeginDocument{\renewcommand\contentsname{Περιεχόμενα}}

\hypersetup{
  colorlinks=true,
  linkcolor=black,
}
\title{
  \includegraphics[height=3cm]{auth.png}\\
  \large{Αριστοτέλειο Πανεπιστήμιο Θεσσαλονίκης}\\
  \large{Τμήμα Ηλεκτρολόγων Μηχανικών Και Μηχανικών Υπολογιστών}\\
  \vspace{2cm} 
  \LARGE{Τεχνολογία Ήχου Και Εικόνας\\Κατηγοριοποίηση Καρδιακών Ήχων} 
  \vspace{3cm} 
}

\author{
  Μουστάκας Γεώργιος 9365\\
  Σαρρής Αναστάσιος Λουκάς 9451\\
  Στεφανίδης Ιωάννης 9587\\
  Σφυράκης Εμμανουήλ 9507
  \vspace{3cm}
}

\date{\today}
\newpage 

\begin{document}

\maketitle

\newpage
\tableofcontents

% Εισαγωγή
\documentclass[../main.tex]{subfiles}
\graphicspath{{\subfix{../images/}}}

\begin{document}

Οι καρδιοπάθειες ταλαιπωρούν, και σε πολλές περιπτώσεις οδηγούν στον θάνατο, ένα
μεγάλο μέρος του πληθυσμού παγκοσμίως. Σύμφωνα με τον Παγκόσμιο Οργανισμό
Υγείας\cite{who} είναι υπεύθυνες για το 30\% των θάνατων ανά έτος. Στις εύπορες
κοινωνίες όπου υπάρχουν τα μέσα και οι πόροι η μαγνητική ακτινογραφία και η
εξέταση με υπέρηχο έχουν αντικαταστήσει την διάγνωση κάποιου καρδιακού
προβλήματος από μια απλή στηθοσκόπιση του ασθενούς. Όμως στις υποανάπτυκτες
χώρες του πλανήτη η κατάσταση δεν είναι η ίδια, καθώς η έλλειψη εξειδικευμένου
προσωπικού ή ακόμα και γιατρών σε κάποιες περιπτώσεις οδηγεί σε μη έγκαιρη
διάγνωσή ενός καρδιακού προβλήματος έχοντας ως αποτέλεσμα τα προαναφερόμενα.

Μία λύση στο πρόβλημα αυτό προσπάθησε να δώσει η πρόκληση του PhysioNet το 2016
\cite{clifford2016classification}. Μέσα απο αυτήν, οι υπεύθυνοι ήθελαν να
ενθαρρύνουν την ανάπτυξη αλγορίθμων κατηγοριοποίησης των καρδιακών ήχων καθώς
επίσης και την δημιουργία μιας μεγάλης δημόσιας βάσης δεδομένων με καρδιακούς
ήχους. Κατάφεραν τη συλλογή 4430 ηχογραφήσεων από 1072 άτομα και 233.512
φωνοκαρδιογραφήματα σε όλο τον πλανήτη.  Τα δείγματα αφορούν τόσο υγιείς όσο και
καρδιοπαθείς ασθενείς που πάσχουν από διάφορες παθήσεις όπως στεφανιαία νόσο ή
πάθηση στις βαλβίδες. Οι ηχογραφήσεις προέρχονται από ετερογενείς πηγές τόσο από
κλινικό εξοπλισμό όσο και από επισκέψεις ιατρών στο σπίτι. Ακόμη παρέχονται
πληροφορίες για το κάθε δείγμα όπως ηλικία, φύλλο,αριθμός ηχογραφήσεων ανά
ασθενή,διάρκεια και περιοχή ηχογράφησης. Τα φωνοκαρδιογραφήματα προέρχονται από
διαφορετικά μέρη του σώματος των ασθενών με τα τέσσερα πιο συχνά σημεία να είναι
οι περιοχές των βαλβίδων(αορτική,μιτροειδής,τριγλωχίνα,πνευμονική). Ένα ποσοστό
δεδομένων έχουν αρκετό θόρυβο ώστε να είναι ρεαλιστική η βάση δεδομένων.  Ο
σκοπός είναι μέσα από ένα μικρό δείγμα ήχου μερικών δευτερολέπτων έως και μερικά
λεπτά, μέσω του αλγορίθμου,να διαχωρίζεται ο ήχος σε φυσιολογικό ή μη
φυσιολογικό, οπότε και χρειάζεται να διαγνωστεί από κάποιον ειδικό. Θα μπορούσε
να χρησιμοποιηθεί ώς κάποια εφαρμογή για κινητό ή ακόμα ως ιστοσελίδα όπου θα
ανεβαίνει ένα ηχητικό αρχείο και θα ξεκινάει ο αλγόριθμος κατηγοριοποίησης.
Αρκετές απόπειρες έχουν γίνει για την σχεδίαση τέτοιων αλγορίθμων στηριζόμενοι
είτε σε εξαγωγή δεδομένων και μέσω μηχανκής μάθησης να γίνει ο διαχωρισμός είτε
με τη χρήση νευρωνικών δικτύων. Εμείς θα προσπαθήσουμε να εξάγουμε τα
αποτελέσματα μας με τον δεύτερο τρόπο όπως φαινεται αναλύτικότερα στην ενότητα
\ref{feature_extraction}.

\end{document}


% Καρδιακοί Ήχοι
\section{Καρδιακοί ήχοι}
Κατά τη διάρκεια του καρδιακού κύκλου,η καρδιά παράγει ηλεκτρική δραστηριότητα,η
οποία στην συνέχεια προκαλεί κολπικές και κοιλιακές συσπάσεις.Αυτό με την σειρά
του οδηγεί το αίμα γύρω από το σώμα.Το άνοιγμα και το κλείσιμο των καρδιακών
βαλβίδων σχετίζεται με επιταχύνσεις-επιβραδύνσεις του αίματος,προκαλώντας
δονήσεις ολόκληρης της καρδιακής δομής.Αυτές οι δονήσεις ακούγονται στα θωρακικά
τοιχώματα και η ακρόαση συγκεκριμένων καρδιακών ήχων μπορεί να δώσει μια ένδειξη
για την υγεία της καρδιάς.Το φωνοκαρδιογράφημα ( PCG) είναι η γραφική
αναπαράσταση μιας εγγραφής καρδιακού ήχου.Ένα τυπικό PCG φαίνεται στην παρακάτω
εικόνα:

\begin{figure}[H]
	\includegraphics[width=\textwidth]{pcg.png}
	\caption{PCG και ECG}
	\label{PCG}
\end{figure}

Όπως φαίνεται και στην εικόνα ένας πλήρης καρδιακός κύκλος στο
φωνοκαρδιογράφημα αποτελείται από τέσσερις διακριτές περιοχές.Αυτές είναι οι
S1, S2, συστολή και διαστολή.Και οι τέσσερις ήχοι που αποτελούν ένα κύκλο
σχετίζονται με το κλείσιμο συγκεκριμένων βαλβίδων και την ροή αίματος από και
πρός τις κοιλίες.



% Η προσέγγιση μας
\section{Η προσέγγισή μας}
\label{our_approach}

Η προσέγγισή μας έχει 3 βασικά μέρη:
\begin{itemize}
    \item \textbf{Κατάτμηση} των PCG δειγμάτων στις βασικές περιοχές του καρδιακού
          ήχου.
    \item \textbf{MFCC μετασχηματισμό} των PCG δειγμάτων σε αναπαράσταση
          χρόνου-συχνότητας της κατανομής της ενέργειας του σήματος.
    \item \textbf{Εκπαίδευση \& Κατηγοροποίηση} των MFCC heat maps με την χρήση
          του συνελικτικού νευρωνικού δικτύου.
\end{itemize}


% % Εργαλεία
% \section{Εργαλεία}
Στην προσπάθεια μας αυτή,η γλώσσα προγραμματισμού με την οποία θα δουλέψουμε
είναι η Python .Η βασική βιβλιοθήκη λογισμικού που θα χρησιμοποιήσουμε είναι το
Tensor Flow.Πρόκειται, για μια συμβολική βιβλιοθήκη μαθηματικών και
χρησιμοποιείται, επίσης, για εφαρμογές μηχανικής μάθησης, όπως νευρωνικά δίκτυα.
Επίσης, σε κάποια σημεία θα εργαστούμε και με το Matlab.


% % Αποτελέσματα
% \documentclass[../main.tex]{subfiles}
\graphicspath{{\subfix{../images/}}}

\begin{document}

Στα παρακάτω σχήματα φαίνονται τρις σημαντικές μετρικές που καταγράφαμε κατά την
διάρκεια της εκπαίδευσης.

Πρώτη είναι η μετρική \textbf{Accuracy} που μας δείχνει πόσο συχνά οι προβλέψεις
του νευρωνικού δικτύου είναι σωστές.

Δεύτερη είναι η μετρική \textbf{Loss} και πιο συγκεκριμένα στα σχήματα
\ref{mfcc_loss} και \ref{spectogram_recall} χρησιμοποιήσαμε την \textit{Binary
	crossentropy loss function} η οποία υπολογίζει πόσο απέχει η πρόβλεψη του
νευρωνικού από την πραγματική τιμή μέσω της παρακάτω συνάρτησης
\ref{eq:binary_crossentropy}.

\begin{equation}\label{eq:binary_crossentropy}
	\mathrm{Loss} = - \frac{1}{\mathrm{output \atop size}} \sum_{i=1}^{\mathrm{output \atop size}} y_i \cdot \mathrm{log}\; {\hat{y}}_i + (1-y_i) \cdot \mathrm{log}\; (1-{\hat{y}}_i)
\end{equation}

Τελευταία μετρική είναι η \textbf{Recall} την οποία θεωρήσαμε πολύ σημαντική για
το συγκεκριμένο πρόβλημα, καθώς μας δείχνει το ποσοστό των σωστά
κατηγοροποιημένων ατόμων με πρόβλημα, από το σύνολο όλων αυτών τον ατόμων
(ιδανικά θα θέλαμε να έχουμε 100\% recall, δηλαδή να μπορούμε να εντοπίσουμε
κάθε άτομο με καρδιακό πρόβλημα).

\subsection{Χρήση MFCC}

Με την χρήση των MFCC ως είσοδο στο νευρωνικό το accuracy σταθεροποιείται κοντά
στο 81\%, ενώ το validation\_loss έχει αρκετή απόκλιση από το loss και κυμαίνετε
ανάμεσα στο 0.4 με 0.5. Το recall έχει δραματικές μεταβολές από εποχή σε εποχή
κάτι που θα εξηγηθεί παρακάτω, με μια μέση τιμή 45\% μετά την
40\textsuperscript{στη} εποχή.

\begin{figure}[H]
	\center
	\includegraphics[width=\textwidth]{../images/mfcc_accuracy.png}
	\caption{Accuracy νευρωνικού δικτύου ανά εποχή με χρήση \textbf{MFCCs} ως
		είσοδο}
	\label{mfcc_accuracy}
\end{figure}
\begin{figure}[H]
	\center
	\includegraphics[width=\textwidth]{../images/mfcc_loss.png}
	\caption{Loss νευρωνικού δικτύου ανά εποχή με χρήση \textbf{MFCCs} ως
		είσοδο}
	\label{mfcc_loss}
\end{figure}
\begin{figure}[H]
	\center
	\includegraphics[width=\textwidth]{../images/mfcc_recall.png}
	\caption{Recall νευρωνικού δικτύου ανά εποχή με χρήση \textbf{MFCCs} ως
		είσοδο}
	\label{mfcc_recall}
\end{figure}


\subsection{Χρήση Mel spectogram}

Όταν χρησιμοποιήσαμε τα mel σπεκτογράμματα ως είσοδο στο νευρωνικό δίκτυο,
είδαμε ελαφρώς καλύτερα αποτελέσματα. Αφού πλέον η ακρίβεια σταθεροποιείται στο
86\% και το validation\_loss δεν απέχει τόσο από το loss με μέση τιμή κάτω από
0.35. Όσο αναφορά το recall, έχουμε κι εδώ μεγάλες μεταβολές από εποχή σε εποχή
με την τιμή να βρίσκετε γύρω από το 57\%.

\begin{figure}[H]
	\center
	\includegraphics[width=\textwidth]{../images/spectogram_accuracy.png}
	\caption{Accuracy νευρωνικού δικτύου ανά εποχή με χρήση \textbf{spectogram} ως
		είσοδο}
	\label{spectogram_accuracy}
\end{figure}
\begin{figure}[H]
	\center
	\includegraphics[width=\textwidth]{../images/spectogram_loss.png}
	\caption{Loss νευρωνικού δικτύου ανά εποχή με χρήση \textbf{spectogram} ως
		είσοδο}
	\label{spectogram_loss}
\end{figure}
\begin{figure}[H]
	\center
	\includegraphics[width=\textwidth]{../images/spectogram_recall.png}
	\caption{Recall νευρωνικού δικτύου ανά εποχή με χρήση \textbf{spectogram} ως
		είσοδο}
	\label{spectogram_recall}
\end{figure}

\subsection{Συμπεράσματα}

Όπως είδαμε παραπάνω η καλύτερη προσπάθεια μας είχε 86\% accuracy με 57\% recall
όμως ακόμα κι αυτή η προσπάθεια δεν είναι ικανοποιητική λόγω των άνισων
δεδομένων. Τα δείγματα που παρείχε η Physionet \cite{physionet} ήταν συνολικά
3228 από τα οποία μόνο τα 665 (20\%) ήταν από ανθρώπους με καρδιοπάθεια.
Αυτό σημαίνει ότι ακόμα κι ένα νευρωνικό που θα κατηγοροποιούσε όλα τα δείγματα
ως 0 (δηλαδή χωρίς καρδιοπάθεια) θα είχε accuracy περίπου 80\% αλλά με 0\%
recall. Αυτή είναι και η αιτία που βλέπουμε και τις μεγάλες μεταβολές από εποχή
σε εποχή στο recall καθώς το δίκτυο έχει την τάση να κατηγοριοποιεί κάθε δείγμα
ως άτομο χωρίς καρδιοπάθεια λόγω της αριθμητικής υπεροχής αυτών τον ατόμων στα
δείγματα.

\end{document}


% % Συζήτηση
% \include{discuss}

\bibliographystyle{ieeetran}
\bibliography{references}


\end{document}
