\section{Εισαγωγή}
Οι καρδιοπάθειες ταλαιπωρούν, και σε πολλές περιπτώσεις οδηγούν στον θάνατο, ένα
μεγάλο μέρος του πληθυσμού παγκοσμίως. Σύμφωνα με τον Παγκόσμιο Οργανισμό Υγείας είναι
υπεύθυνες για το 30\% των θάνατων ανά έτος. Στις εύπορες κοινωνίες όπου υπάρχουν τα μέσα
και οι πόροι η μαγνητική ακτινογραφία και η εξέταση με υπέρηχο έχουν αντικαταστήσει την
διάγνωση κάποιου καρδιακού προβλήματος από μια απλή στηθοσκόπιση του ασθενούς. Όμως στις
υποανάπτυκτες χώρες του πλανήτη η κατάσταση δεν είναι η ίδια, καθώς η έλλειψη εξειδικευμένου
προσωπικού ή ακόμα και γιατρών σε κάποιες περιπτώσεις οδηγεί σε μη έγκαιρη διάγνωσή ενός καρδιακού
προβλήματος έχοντας ως αποτέλεσμα τα προαναφερόμενα.

Μία λύση στο πρόβλημα αυτό προσπάθησε να δώσει η πρόκληση του PhysioNet το
2016. Μέσα απο αυτήν, οι υπεύθυνοι ήθελαν να ενθαρρύνουν την ανάπτυξη αλγορίθμων
κατηγοριοποίησης των καρδιακών ήχων καθώς επίσης και την δημιουργία μιας μεγάλης
δημόσιας βάσης δεδομένων με καρδιακούς ήχους. Κατάφεραν τη συλλογή 4430 ηχογραφήσεις
από 1072 άτομα και 233.512 φωνοκαρδιογραφήματα σε όλο τον πλανήτη. Τα δείγματα αφορούν
τόσο υγιείς όσο και καρδιοπαθείς ασθενείς που πάσχουν από διάφορες παθήσεις όπως στεφανιαία
νόσο ή πάθηση στις βαλβίδες. Οι ηχογραφήσεις προέρχονται από ετερογενείς πηγές τόσο από κλινικό
εξοπλισμό όσο και από επισκέψεις ιατρών στο σπίτι. Ακόμη παρέχονται πληροφορίες για το κάθε δείγμα
όπως ηλικία, φύλλο,αριθμός ηχογραφήσεων ανά ασθενή,διάρκεια και περιοχή ηχογράφησης. Τα φωνοκαρδιογραφήματα
προέρχονται από διαφορετικά μέρη του σώματος των ασθενών με τα τέσσερα
πιο συχνά σημεία να είναι οι περιοχές των βαλβίδων(αορτική,μιτροειδής,τριγλωχίνα,πνευμονική). Ένα
ποσοστό δεδομένων έχουν αρκετό θόρυβο ώστε να είναι ρεαλιστική η βάση δεδομένων.
Ο σκοπός είναι μέσα από ένα μικρό δείγμα ήχου μερικών δευτερολέπτων έως και μερικά λεπτά,
μέσω του αλγορίθμου,να διαχωρίζεταιο ήχος σε φυσιολογικό ή μη φυσιολογικό, οπότε και χρειάζεται
να διαγνωστεί από κάποιον ειδικό. Θα μπορούσε να χρησιμοποιηθεί ώς κάποια εφαρμογή για κινήτο
ή ακόμα ως ιστοσελίδα όπου θα ανεβαίνει ένα ηχητικό αρχείο και θα ξεκινάει ο αλγόριθμος κατηγοριοποίησης.
Αρκετές απόπειρες έχουν γίνει για την σχεδίαση τέτοιων αλγορίθμων στηριζόμενοι είτε σε εξαγωγή δεδομένων
και μέσω μηχανκής μάθησης να γίνει ο διαχωρισμός είτε με τη χρήση νευρωνικών δικτύων. Εμείς
θα προσπαθήσουμε να εξάγουμε τα αποτελέσματα μας με τον δεύτερο τρόπο όπως φαινεται αναλύτικότερα
στο Κεφάλαιο \ref{our_approach}.
