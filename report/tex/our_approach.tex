\section{Η προσέγγισή μας}\label{our_approach}

Η προσέγγισή μας έχει 3 βασικά μέρη:
\begin{itemize}
	\item \textbf{Κατάτμηση} των PCG δειγμάτων στις βασικές περιοχές του καρδιακού
	      ήχου.
	\item \textbf{MFCC μετασχηματισμό} των PCG δειγμάτων σε αναπαράσταση
	      χρόνου-συχνότητας της κατανομής της ενέργειας του σήματος.
	\item \textbf{Εκπαίδευση \& Κατηγοροποίηση} των MFCC heat maps με την χρήση
	      του συνελικτικού νευρωνικού δικτύου.
\end{itemize}

\subsection{Κατάτμηση Δειγμάτων}

Για να κάνουμε την κατάτμηση των PCG δειγμάτων όπως φαίνεται στην εικόνα
\ref{PCG} θα χρησιμοποιήσουμε των αλγόριθμο του Springer
\cite{springer2015logistic} τον οποίο τον παρείχε ο διαγονισμός στους
συμμετέχοντες. Ωστόσο δεν θα χρησιμοποιήσουμε όλα τα δεδομένα που μας δίνει ο
αλγόριθμος, αλλά αυτό που θα κάνουμε είναι να βρίσκουμε που ξεκινάει το πρώτο
\emph{S1} και στη συνέχεια θα αναλύουμε τα 3 επόμενα δευτερόλεπτα. Αυτή η
διαδικασία θα γίνεται ώστε τα δείγματα με τα οποία θα εκπαιδεύσουμε το νευρονικό
δίκτυο να είναι ``ευθυγραμισμένα'' μεταξύ τους.


\subsection{Μετασχησματισμός MFCC}
\subsection{Νευρωνικό δίκτυο}
\subsubsection{Αρχιτεκτονική δικτύου}
