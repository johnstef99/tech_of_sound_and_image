\section{Καρδιακοί ήχοι}
Κατά τη διάρκεια του καρδιακού κύκλου,η καρδιά παράγει ηλεκτρική δραστηριότητα,η
οποία στην συνέχεια προκαλεί κολπικές και κοιλιακές συσπάσεις.Αυτό με την σειρά
του οδηγεί το αίμα γύρω από το σώμα.Το άνοιγμα και το κλείσιμο των καρδιακών
βαλβίδων σχετίζεται με επιταχύνσεις-επιβραδύνσεις του αίματος,προκαλώντας
δονήσεις ολόκληρης της καρδιακής δομής.Αυτές οι δονήσεις ακούγονται στα θωρακικά
τοιχώματα και η ακρόαση συγκεκριμένων καρδιακών ήχων μπορεί να δώσει μια ένδειξη
για την υγεία της καρδιάς.Το φωνοκαρδιογράφημα ( PCG) είναι η γραφική
αναπαράσταση μιας εγγραφής καρδιακού ήχου.Ένα τυπικό PCG φαίνεται στην παρακάτω
εικόνα:

\begin{figure}[H]
	\includegraphics[width=\textwidth]{pcg.png}
	\caption{PCG και ECG}
	\label{PCG}
\end{figure}

Όπως φαίνεται και στην εικόνα ένας πλήρης καρδιακός κύκλος στο
φωνοκαρδιογράφημα αποτελείται από τέσσερις διακριτές περιοχές.Αυτές είναι οι
S1, S2, συστολή και διαστολή.Και οι τέσσερις ήχοι που αποτελούν ένα κύκλο
σχετίζονται με το κλείσιμο συγκεκριμένων βαλβίδων και την ροή αίματος από και
πρός τις κοιλίες.

